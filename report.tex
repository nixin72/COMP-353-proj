\documentclass{article}
\usepackage[utf8]{inputenc}
\usepackage{tikz}
\usepackage{graphicx}
\usepackage{listings}
\usepackage{color}

\usetikzlibrary{er,positioning}

\title{COMP-353: Warmup Project Documentation}
\author{Insert Team Members}
\date{February 2020}

\begin{document}

\maketitle

\section{(1) DESIGN:}
 The E/R diagram of the design of the database given in the project description (or a revised version, if deemed necessary).

\includegraphics[width=15cm,keepaspectratio]{ER.png}

\pagebreak

\section{(2) The SQL statements formulated and used to create the database.}

\definecolor{dkgreen}{rgb}{0,0.6,0}
\definecolor{gray}{rgb}{0.5,0.5,0.5}
\definecolor{mauve}{rgb}{0.58,0,0.82}

\lstset{frame=tb,
  language=sql,
  aboveskip=3mm,
  belowskip=3mm,
  showstringspaces=false,
  columns=flexible,
  basicstyle={\small\ttfamily},
  numbers=none,
  numberstyle=\tiny\color{gray},
  keywordstyle=\color{blue},
  commentstyle=\color{dkgreen},
  stringstyle=\color{mauve},
  breaklines=true,
  breakatwhitespace=true,
  tabsize=3
}

\begin{lstlisting}
CREATE TABLE `match` (
  mid        INT AUTO_INCREMENT,
  date_time  DATETIME NOT NULL,
  stadium    VARCHAR(255) NOT NULL,
  home_tid   INT,
  home_goals INT,
  away_tid   INT,
  away_goals INT,
  PRIMARY KEY(mid)
);

CREATE TABLE team (
  tid        INT AUTO_INCREMENT,
  name       VARCHAR(255) NOT NULL,
  city       VARCHAR(255),
  captain_id INT,
  PRIMARY KEY(tid)
);

CREATE TABLE player (
  pid         INT AUTO_INCREMENT,
  name        VARCHAR(255) NOT NULL,
  position    VARCHAR(255),
  dob         DATE,
  current_tid INT,
  start_date  DATE NOT NULL,
  PRIMARY KEY(pid)
);

CREATE TABLE played_in (
  tid     INT,
  mid     INT,
  pid     INT,
  goals   INT,
  minutes INT,
  PRIMARY KEY(tid, mid, pid)
);

CREATE TABLE previous_teams (
  tid        INT,
  pid        INT,
  start_date DATE,
  end_date   DATE,
  PRIMARY KEY(tid, pid, start_date)
);

ALTER TABLE `match`
  ADD FOREIGN KEY(home_tid) REFERENCES team(tid),
  ADD FOREIGN KEY(away_tid) REFERENCES team(tid);

ALTER TABLE team
  ADD FOREIGN KEY(captain_id) REFERENCES player(pid);

ALTER TABLE player
  ADD FOREIGN KEY(current_tid) REFERENCES team(tid);

ALTER TABLE played_in
  ADD FOREIGN KEY(tid) REFERENCES team(tid),
  ADD FOREIGN KEY(mid) REFERENCES `match`(mid),
  ADD FOREIGN KEY(pid) REFERENCES player(pid);

ALTER TABLE previous_teams
  ADD FOREIGN KEY(tid) REFERENCES team(tid),
  ADD FOREIGN KEY(pid) REFERENCES player(pid);

\end{lstlisting}

\section{(3) The SQL statements formulated to express the required queries and transactions mentioned.}

1)
\begin{lstlisting}
SELECT dob, position
FROM player
WHERE LEFT(name, 1) = 'T';
\end{lstlisting}{}

2)
\begin{lstlisting}
SELECT player.name
FROM player, team
WHERE current_tid = tid AND tid = 2;
\end{lstlisting}{}

3)
\begin{lstlisting}
SELECT player.name, pid, captain_id
FROM player, team
WHERE pid = captain_id AND LEFT(player.name, 1) = 'J';
\end{lstlisting}{}

\pagebreak

4)
\begin{lstlisting}
SELECT date_time, stadium
FROM `match`
WHERE home_goals = away_goals
\end{lstlisting}{}

5) TO FIX
\begin{lstlisting}
SELECT COUNT(goals)
FROM played_in, match
WHERE date_time LIKE '2019%';
\end{lstlisting}{}

6)
\begin{lstlisting}
SELECT DISTINCT name, player.pid
FROM player, played_in
WHERE goals = 0 AND player.pid = played_in.pid;
\end{lstlisting}{}

7)
\begin{lstlisting}
SELECT name, player.pid, goals
FROM player, played_in;
\end{lstlisting}{}

\section{(4) Populate each table in the database with at least 10 representative and appropriate tuples. }

\section{(5) For each relation R created in your database, report the result of the following SQL statement:}


\end{document}
